\documentclass[12pkt,a4paper]{article}
\usepackage{graphicx}
\graphicspath{pictures/}


\title{\textbf{Chapter 1}}
\author{\textit{Ola Ignacyk}}
\date{\underline{November 2022}}


\begin{document}
\maketitle

\begin{abstract}
    Jest to rozdział stworzony przeze mnie b zrozumieć podstawy LaTeX.
\end{abstract}


\section{ Wyrażenie matematyczne:}
Wzór Pitagorasa ma postać:
 \[a^2+b^2=c^2\]

 \section{Zdjęcie:}

\begin{center}
   \includegraphics[scale 2]{barbie.jpg} 
    \label{fig:zdjecie}
\end{center}


\section{Tabela:}
\begin{table}[h!]
\centering
\begin{tabular}{||c c c c||} 
 \hline
 1 & 2 & 3 & 4 \\ [0.5ex] 
 \hline\hline
 - & - & - & - \\ 
 - & - & - & - \\
 - & - & - & - \\
 - & - & - & - \\
 - & - & - & - \\ [1ex] 
 \hline
\end{tabular}
\label{table:tabela}
\end{table}

\section{Lista numerowana:}
\begin{enumerate}
    \item Po pierwsze...
    \item Po drugie...
\end{enumerate}


\section{Lista nienumerowana:}
\begin{itemize}
    \item Pierwsze:
    \item Drugie:
\end{itemize}

\section{Krótki tekst:}
- \emph{Jutro przyniosę film} - ogłosił. - \emph{I macie mi na nim nie zasnąć, bo inaczej będziecie musieli odegrać go sami.}

Kilka osób zaśmiało się, słysząc tę \textbf{groźbę}, ale większość już się oddaliła, \underline{jeśli nie ciałem, to przynajmniej duchem.}

Tak to się zaczęło. Kiedy jest się Światłem, dzień i noc mają mniejsze znaczenie. Noc nie jest potrzebna do odpoczynku - to tylko kilka irytujących godzin ciemności.\\ Ten łańcuch dni i nocy pomaga jednak Żywym odmierzać swoją podróż. To historia mojej podróży z powrotem do świata Żywych. W ciągu niespełna tygodnia miałam znów zamieszkać w ludzkim ciele.

Który akapit to tabela:
\ref{table:tabela}

Który akapit to zdjęcie:
\ref{fig:zdjecie}


\end{document}

