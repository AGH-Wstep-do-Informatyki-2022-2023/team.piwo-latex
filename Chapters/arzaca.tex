\documentclass[12pt]{article}
\usepackage{graphicx}
\usepackage[utf8]{inputenc}
\graphicspath{{./pictures}}

\title{Tytuł}
\author{Aleksandra Rząca}
\date{November 2022}

\begin{document}

\maketitle

\section{Wyrażenie matematyczne:}
    \large\[x^2+y^3+z^4=0\]

\section{Zdjęcie:}
\begin{center}
    \includegraphics[scale=0.25]{Pictures/kamyk.jpg}
    \\
    \caption{Magiczny kamień}
    \label{fig:kamyk}
\end{center}

\section{Listy:}
\begin{enumerate}
    \item One
    \item Two
    \item Three
\end{enumerate}

\vspace{2mm}

\begin{itemize}
    \item Hello
    \item Hi
    \item Welcome
\end{itemize}

\section{Tekst:}
Gdy \textbf{Gregor Samsa} obudził się pewnego rana z niespokojnych snów, stwierdził, że zmienił się w łóżku w potwornego robaka.

Leżał na \underline{grzbiecie} twardym jak \emph{pancerz}, a kiedy uniósł nieco głowę, widział swój sklepiony, brązowy, podzielony sztywnymi łukami brzuch, na którym ledwo mogła utrzymać się całkiem już ześlizgująca się kołdra. Liczne, w porównaniu z dawnymi rozmiarami, żałośnie cienkie nogi migały mu bezradnie przed oczami.

\section{Tabela:}
\begin{table}[h!]
    \centering
    \begin{tabular}{|c|c|c|c|c|}
    \hline
    * & 2 & 3 & 4 & 5 \\ \hline
    2 & 4 & 1 & 3 & 0 \\ \hline
    \end{tabular}
\end{table}

\end{document}

